\section{Заключение}
\label{sec:Chapter6} \index{Chapter6}

Таким образом, в данной работе было проведено исследование проблемы переноса разметки требований между многоверсионными текстовыми документами в рамках инструмента управления требованиями Requality. Было изучено решение, использующееся в Requality на данный момент, и разработан альтернативный алгоритм, основанный на предположении об отсутствии изменений названий и структуры разделов в новой версии спецификации требований.

Была написана реализация алгоритма на языке программировния Java, и проведены эксперименты на нескольких документах с размеченными требованиями, а также сравнение по эффективности с реализацией алгоритма, использующегося в Requality.

\subsection{Возможные улучшения}

Заметим, что приведенный в тексте данной работы алгоритм легко обобщается на случай поиска в тексте раздела нечеткого соответствия тексту объединения фрагментов. Для этого достаточно заменить алгоритм прямого поиска в строке на один из алгоритмов нечеткого поиска строки в тексте, использующий методы, описанные, например, в \cite{web:StrNotExact}. Однако отказаться от используемого предположения об отсутствии изменений названий и иерархии разделов избавиться для данного алгоритма не представляется возможным.